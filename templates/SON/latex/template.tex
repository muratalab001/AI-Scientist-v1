\documentclass{article}
\usepackage{iclr2024_conference,times}
\usepackage[utf8]{inputenc}
\usepackage[T1]{fontenc}
\usepackage{hyperref}
\usepackage{url}
\usepackage{booktabs}
\usepackage{amsfonts}
\usepackage{nicefrac}
\usepackage{microtype}
\usepackage{titletoc}
\usepackage{subcaption}
\usepackage{graphicx}
\usepackage{amsmath}
\usepackage{multirow}
\usepackage{color}
\usepackage{colortbl}
\usepackage{cleveref}
\usepackage{algorithm}
\usepackage{algorithmicx}
\usepackage{algpseudocode}
\DeclareMathOperator*{\argmin}{arg\,min}
\DeclareMathOperator*{\argmax}{arg\,max}
\graphicspath{{../}}
\begin{filecontents}{references.bib}
@article{lu2024aiscientist,
  title={The {AI} {S}cientist: Towards Fully Automated Open-Ended Scientific Discovery},
  author={Lu, Chris and Lu, Cong and Lange, Robert Tjarko and Foerster, Jakob and Clune, Jeff and Ha, David},
  journal={arXiv preprint arXiv:2408.06292},
  year={2024}
}
@article{mezzadri2007generate,
  title={How to generate random matrices from the classical compact groups},
  author={Mezzadri, Francesco},
  journal={Notices of the AMS},
  volume={54},
  number={5},
  pages={592--604},
  year={2007}
}
\end{filecontents}
\title{Statistical Properties of Random SO(N) Matrices}
\author{LLM\\Department of Computer Science\\University of LLMs\\}
\newcommand{\fix}{\marginpar{FIX}}
\newcommand{\new}{\marginpar{NEW}}
\begin{document}
\maketitle
\begin{abstract}
We investigate the statistical properties of random matrices drawn from the special orthogonal group SO(N). We analyze the empirical distributions of traces and determinants for various N, and compare the results to theoretical expectations.
\end{abstract}
\section{Introduction}
\label{sec:intro}
Random matrix theory plays a central role in mathematical physics and statistics. In this work, we focus on the special orthogonal group SO(N) and study the statistical properties of random matrices generated from this group.
\section{Related Work}
\label{sec:related}
Previous studies have analyzed random matrices from classical groups such as U(N), SU(N), and SO(N) \citep{mezzadri2007generate}.
\section{Background}
\label{sec:background}
The special orthogonal group SO(N) consists of all N x N real orthogonal matrices with determinant +1. Random SO(N) matrices can be generated using the Haar measure.
\section{Method}
\label{sec:method}
We generate random SO(N) matrices using the QR decomposition method, ensuring the determinant is +1. For each matrix, we compute the trace and determinant.
\section{Experimental Setup}
\label{sec:experimental}
We generate 1000 random SO(N) matrices for various values of N (e.g., N=3,5,10). We analyze the empirical distributions of traces and determinants.
\section{Results}
\label{sec:results}
We present histograms of the trace and determinant distributions. The determinant is always +1 up to numerical precision, while the trace distribution depends on N.
\begin{figure}[h]
    \centering
    \includegraphics[width=0.8\textwidth]{trace_hist_run_0.png}
    \caption{Histogram of traces for 1000 random SO(N) matrices (N=5).}
    \label{fig:trace_hist}
\end{figure}
\section{Conclusions and Future Work}
\label{sec:conclusion}
We have demonstrated the empirical properties of random SO(N) matrices. Future work includes analytical derivation of trace distributions and extension to other groups.
This work was generated by \textsc{The AI Scientist} \citep{lu2024aiscientist}.
\bibliographystyle{iclr2024_conference}
\bibliography{references}
\end{document}
