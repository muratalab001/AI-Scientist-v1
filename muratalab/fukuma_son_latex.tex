%%%%%%%%%%%%%%%%%%%%%%%%%%%%%%%%%%%%%%%%%%%%%%%%%%%%%%%
%%%%%%%%%%%%%%%%%%%%%%%%%%%%%%%%%%%%%%%%%%%%%%%%%%%%%%%
\section{Random generation of elements of $\mathfrak{so}(n)$ and its complexification}
\label{sec:Gaussian_SOn}
%%%%%%%%%%%%%%%%%%%%%%%%%%%%%%%%%%%%%%%%%%%%%%%%%%%%%%%
%%%%%%%%%%%%%%%%%%%%%%%%%%%%%%%%%%%%%%%%%%%%%%%%%%%%%%%

Let $\{T_a\}$ $(a = 1,\ldots, N\,(=\dim \mathfrak{so}(n)))$ be a basis of the Lie algebra $\mathfrak{so}(n)$, satisfying
$T_a^T = -T_a$ and $-\tr (T_a T_b) = \delta_{ab}$.
We present algorithms for generating random elements $X = T_a x^a \in \mathfrak{so}(n)$ $(x^a \in \mathbb{R})$ according to the Gaussian distribution $e^{-\langle X,X\rangle/(2\sigma^2)} = e^{-(x^a)^2/(2\sigma^2)}$,
\begin{align}
  \langle x^a \rangle = 0,\quad \langle x^a x^b \rangle = \sigma^2\,\delta^{ab},
\end{align}
and also for generating random element $Z = T_a z^a \in \mathfrak{so}(n)_\mathbb{C}$ $(z^a \in \mathbb{C})$ according to the Gaussian distribution $e^{-\langle Z,Z\rangle/(2\sigma^2)} = e^{- \overline{z^a} z^a/(2\sigma^2)}$,
\begin{align}
  \langle z^a \rangle = 0,\quad \langle z^a z^b \rangle = \langle \overline{z^a} \overline{z^b} \rangle = 0,\quad \langle z^a \overline{z^b} \rangle = 2\sigma^2\,\delta^{ab}.
\end{align}
As in the $SU(n)$ case, a direct algorithm is to generate $x^a$ and $y^a$ ($a=1,\ldots,N$) from the standard Gaussian and set $X = T_a x^a$ and $Z = T_a (x^a + i y^a)$. Below we present an algorithm that does not use the explicit form of $T_a$.

%%%%%%%%%%%%%%%%%%%%%%%%%%%%%%%%%%%%%%%%%%%%%%%%%%%%%%%
\subsection{$G = SO(n)$}
\label{sec:Gaussian_SO(n)}
%%%%%%%%%%%%%%%%%%%%%%%%%%%%%%%%%%%%%%%%%%%%%%%%%%%%%%%

In this case, $N = \dim \mathfrak{so}(n) = n(n-1)/2$. A random variable $X = T_a x^a \in \mathfrak{so}(n) = \mathrm{Lie}\,SO(n)$ satisfying the above Gaussian can be constructed by the following two steps:

\noindent\underline{(1)}: Generate an independent Gaussian random matrix $\xi = (\xi_{ij})$,
\begin{align}
  \xi_{ij} \sim e^{-\xi_{ij}^2/(2\sigma^2)}\quad (i,j = 1,\ldots,n).
\end{align}

\noindent\underline{(2)}: Set $X = (X_{ij})$ as the real antisymmetric projection
\begin{align}
  X = \tfrac{1}{2}\,(\xi - \xi^T),
\end{align}
which manifestly satisfies $X^T = -X$ and hence $X \in \mathfrak{so}(n)$.

To verify that this construction yields the desired distribution, write $x^a = -\tr (T_a X) = \tr (T^a X) = T^a_{ji} X_{ij}$. Then
\begin{align}
  \langle x^a \rangle = T^a_{ji}\,\langle X_{ij} \rangle = 0.
\end{align}
For the second moment,
\begin{align}
  \langle X_{ij} X_{kl} \rangle
  &= \tfrac{1}{4}\,\Bigl\langle (\xi_{ij} - \xi_{ji})(\xi_{kl} - \xi_{lk}) \Bigr\rangle
   = \tfrac{\sigma^2}{2}\,(\delta_{il}\,\delta_{jk} - \delta_{ik}\,\delta_{jl}),
\end{align}
so that
\begin{align}
  \langle x^a x^b \rangle = T^a_{ji} T^b_{lk} \langle X_{ij} X_{kl} \rangle
  = -\sigma^2\,\tr (T^a T^b) = \sigma^2\,\delta^{ab},
\end{align}
as required by the normalization $-\tr(T_a T_b)=\delta_{ab}$.

For the complexified algebra $\mathfrak{so}(n)_\mathbb{C}$, generate two i.i.d. real Gaussian matrices $\xi$ and $\eta$ and set
\begin{align}
  Z = \tfrac{1}{2}(\xi - \xi^T) + \tfrac{i}{2}(\eta - \eta^T),
\end{align}
which distributes the coordinates $z^a$ as independent complex Gaussians with variance $2\sigma^2$.


